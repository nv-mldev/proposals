\chapter{Methodology and Phased Rollout}
\label{chap:methodology}

Our development methodology is designed to maximize efficiency and minimize risk by separating software development from hardware dependency and tackling challenges based on client priorities.

\section{Simulation-First Development}
The entire DAI software framework will first be developed and validated in a simulated environment. This is a critical step that allows for parallel development while the Accenture team procures and sets up the physical hardware.

\subsection{Simulation Environment}
We will utilize industry-standard robotics simulators like \textbf{Gazebo} or \textbf{NVIDIA Isaac Sim}, integrated with the \textbf{Robot Operating System (ROS)}.

\subsection{Benefits}
\begin{itemize}
    \item \textbf{Safety:} All control algorithms and closed-loop actions can be tested in a sandbox, preventing damage to expensive hardware.
    \item \textbf{Speed:} Rapid iteration on algorithms is possible without the overhead of physical setup and reset procedures.
    \item \textbf{Decoupled Timeline:} DAI's software development proceeds independently, ensuring project milestones are met even with hardware delays.
\end{itemize}

\section{Phased Rollout by Client Priority}
Once the framework is validated in simulation, we will deploy it onto the physical hardware in phases, beginning with the client's top priority: the robotic arm.

\subsection{Phase 1: Robotic Arm — High-Precision Manipulation (Months 1-4)}
This phase tackles the most immediate and complex challenge to deliver value early.
\begin{itemize}
    \item \textbf{Months 1-2:} Develop and perfect the Hand-Eye Calibration procedure in simulation, then validate on the lab's physical robot.
    \item \textbf{Months 3-4:} Implement and test high-accuracy pick-and-place routines. Refine tracking algorithms for moving objects on the physical system.
    \item \textbf{Goal:} Deliver a functional and robust high-precision manipulation solution for the robotic arm.
\end{itemize}

\subsection{Phase 2: PIVs \& AGVs — Foundational Mobile Platforms (Months 5-7)}
This phase adapts the core vision framework to constrained mobile platforms.
\begin{itemize}
    \item Develop and test operator-assist features for the PIV.
    \item Implement and refine line/marker-following control loops for the AGV.
    \item \textbf{Goal:} Prove the framework's adaptability to different mobility and interaction models.
\end{itemize}

\subsection{Phase 3: AMRs — Full Autonomy (Months 8-10)}
This phase addresses the most complex platform involving dynamic navigation.
\begin{itemize}
    \item Integrate the vision framework with the AMR's SLAM and navigation stack.
    \item Develop and test intelligent "seek" behaviors and dynamic obstacle avoidance.
    \item \textbf{Goal:} Achieve robust performance in an unconstrained, real-world environment.
\end{itemize}

\subsection{Phase 4: System Integration \& Final Demonstration (Months 11-12)}
The final phase focuses on combined operations and project delivery.
\begin{itemize}
    \item If required, test scenarios involving multiple platforms (e.g., an AMR delivering a part to a robotic arm).
    \item Finalize documentation, performance benchmarking, and prepare the final software framework for handover.
    \item \textbf{Goal:} Deliver a complete, validated, and documented software solution.
\end{itemize}